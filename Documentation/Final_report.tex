% !TeX spellcheck = es_CO
\documentclass[12pt, oneside]{book}              % Book class in 11 points
\usepackage{amsmath, amsthm, amssymb}
\usepackage[spanish,mexico]{babel}
\usepackage[utf8]{inputenc}
\usepackage{cancel}
\usepackage{graphicx}
\setlength{\parskip}{4mm}

% For several graphics per figure
\usepackage{caption} %[caption=false]
\usepackage{subfigure}

% For centering elements in tables
\usepackage{array}

% For multirows and multicolumns in tables
\usepackage{multicol}
\usepackage{multirow}

% Epigraph
\usepackage{epigraph}

%\oddsidemargin 0.4cm \evensidemargin 0.4cm \topmargin 0.0cm
%\textwidth 15.7cm \textheight 22.0cm \headsep 1.2cm \footskip 1.0cm

\title{\bf Documentation of the project: \\ ISR jet tagging}
\author{\textbf{Autor:}\\ Andr\'es Felipe Garc\'ia Albarrac\'in \\ \\ \textbf{Asesor:} \\ Juan Carlos Sanabria, Ph.D.}
\date{\today}

\begin{document}                        % End of preamble, start of text.
	\frontmatter                            % only in book class (roman page #s)
	\maketitle                              % Print title page.
	\tableofcontents                        % Print table of contents
	\mainmatter                             % only in book class (arabic page #s)

\chapter{Introduction}

During the last semester of 2014, I did my Undergraduate Thesis Project entitled ``\textit{Design of algorithms to 
identify high momentum Initial State Radiation (ISR) Jets in proton – proton collision events}'', under the supervision 
of Juan Carlos Sanabria, Ph.D.. As the name suggests, the project consisted in the proposal of an algorithm to identify
ISR jets. Due to the promising results, I was employed during the first semester of 2015 under the program ``Joven 
Investigador'' of COLCIENCIAS in order to improve the initially obtained results. Throughout this time, several codes
and programs were developed. To encourage the continuation of this project, this report has been written with a 
summary of all the technical work done so far.

In the second chapter, the simulation chain will be presented. This chapter contains the usage of the programs
MadGraph 5.2 \cite{MadGraph}, Pythia 8.2 

\begin{thebibliography}{99}
	\addcontentsline{toc}{chapter}{Bibliograf\'ia}

	\bibitem{MadGraph} 
	  J.~Alwall, R.~Frederix, S.~Frixione, V.~Hirschi, F.~Maltoni, O.~Mattelaer, H.-S.~Shao and T.~Stelzer {\it et al.},
	  \emph{``The automated computation of tree-level and next-to-leading order differential cross sections, and their matching to parton shower simulations,''}
	  JHEP {\bf 1407}, 079 (2014)
	  [arXiv:1405.0301 [hep-ph]].
	  
	\bibitem{Pythia} 
	  T.~Sjöstrand, S.~Ask, J.~R.~Christiansen, R.~Corke, N.~Desai, P.~Ilten, S.~Mrenna and S.~Prestel {\it et al.},
	  \emph{``An Introduction to PYTHIA 8.2,''}
	  Comput.\ Phys.\ Commun.\  {\bf 191}, 159 (2015)
	  [arXiv:1410.3012 [hep-ph]].
	  

\end{thebibliography}
	

\end{document}
